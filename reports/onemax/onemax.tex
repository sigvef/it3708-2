\documentclass{article}
\usepackage[utf8]{inputenc}

\title{Homework Module: \textbf{Programming the Basics of an Evolutionary Algorithm(EA)}}

\author{Sigve Sebastian Farstad}

\begin{document}

\maketitle

\section{Introduction}

This report presents a solution to Homework Module: Programming the Basics of an Evolutionary Algorithm(EA), IT3708, spring 2014, NTNU.
The assignment is to implement all of the basic components of an evolutionary algorithm, and use them to solve the One-Max problem.

\section{The Evolutionary Algorithm Implementation}

The evolutionary algorithm solver presented in this solution is implemented as modular python library.
A generic solver function solves arbitrary problems supplied by the user of the library.
Problems consist of a set of configuration options for selecting 

\section{The One-Max Problem}

The One-Max problem is a simple problem popularly chosen as a demonstration problem for testing evolutionary algorithm implementations.
It has no other real-world applications.
The problem is: what is the binary string of length $ n $ that contains the largest number of Ones?
The problem is trivial, but is good for basic evolutionary algorithm testing because of its transparency and simplicity.

\subsection{Genotype and Phenotype Representation}

A single genotype is represented as a binary vector of length $ n $.
Because of the simplicity of the program, the phenotype is identical.

\subsection{Fitness Evaluation}

The fitness function $ f_{\textsc{OneMax}}(g) $ of a genotype $ g $ is defined as equation \ref{equation:fitness}, where $ g_i $ is the $ i $\textsuperscript{th} bit of the genome $ g $.

\begin{equation}\label{equation:fitness}
f_{\textsc{OneMax}}(g) = \sum_{i=1}^{n} g_i
\end{equation}

\end{document}
